\documentclass{article}
\usepackage[margin=1in]{geometry}
\usepackage{longtable}
\usepackage{booktabs}
\usepackage{hyperref}

\title{Breakdown of \texttt{Connect-Kubernetes} Execution Environment}
\author{}
\date{}

\begin{document}

\maketitle

\section*{Overview}

This document outlines the relationship between the operating systems, execution layers, and components involved when using the \texttt{Connect-Kubernetes} command within the DATA420-25S1 environment.

\section*{Component Breakdown}

\begin{longtable}{@{}lll@{}}
\toprule
\textbf{Component} & \textbf{OS / Layer} & \textbf{Notes} \\
\midrule
PowerShell & Windows (MADS Desktop) & You launch it directly \\
\texttt{Connect-Kubernetes} & PowerShell script & Talks to Kubernetes API \\
Kubernetes Cluster & Linux-based (cloud backend) & Where your Jupyter container runs \\
Jupyter container & Linux (e.g., Ubuntu) & Where Spark/Python executes \\
\bottomrule
\end{longtable}

\section*{Explanation}

The \texttt{Connect-Kubernetes} command is executed within the Windows-based MADS Desktop using PowerShell. This command acts as a client interface to the cloud-hosted Kubernetes cluster. Although the command runs locally on Windows, it connects to a backend system where containers are orchestrated on a Linux-based Kubernetes environment.

The actual Spark and Python computations are carried out within a Linux container, which is started and managed via Kubernetes once \texttt{Start-Jupyter} is called.

\end{document}
