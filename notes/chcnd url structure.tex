\documentclass{article}
\usepackage{booktabs}
\usepackage{geometry}
\usepackage{fontspec}
\usepackage{array}
\setmonofont{Consolas}

% Define a column type with raggedright and no hyphenation
\newcolumntype{L}[1]{>{\raggedright\arraybackslash}p{#1}}

\begin{document}

\section*{Azure Blob Storage URL Structure}

This paragraph is fully hyphenated and should wrap naturally across lines without restriction.

\noindent\texttt{wasbs://campus-data@madsstorage002.blob.core.windows.net/ghcnd/daily/2024.csv.gz}

\bigskip

\noindent
\begin{tabular}{@{}L{0.35\textwidth} L{0.12\textwidth} L{0.50\textwidth}@{}}
\toprule
\textbf{Component} & \textbf{Type} & \textbf{Description} \\
\midrule
\texttt{wasbs://} & Protocol & Azure Blob Storage over SSL (used with Hadoop-compatible connectors) \\
\texttt{campus-data} & Container & Like a top-level directory within the blob storage account \\
\texttt{madsstorage002} & Storage account & The Azure storage account name \\
\texttt{blob.core.windows.net} & Service domain & Azure’s blob service endpoint domain \\
\texttt{/ghcnd/daily/2024.csv.gz} & Path & Virtual directory and file structure within the container \\
\texttt{ghcnd} & Container name & \texttt{<container>} \\
\texttt{daily} & Optional folder(s) within container & \texttt{<directory>}\\
\texttt{2024.csv.gz} & Blob name (file) &  \texttt{<filename>} \\
\bottomrule
\end{tabular}

\bigskip

This paragraph is also hyphenated like any normal LaTeX text. For example, hyphenation is useful when lines are justified.

\end{document}
